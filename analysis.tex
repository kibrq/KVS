\section*{Анализ}
	\subsection*{Предположения}
	\begin{enumerate}
		\item Размер значения и ключа, постоянные.
		
		\item Размер блока не слишком большой (4kB).
		
		\item Ключ и значение оффсета помещаются в блок не менее 100 раз
		
		\item Значение помещается в один блок и может быть считано и записано одной элементарной операцией чтения/записи
		
	\end{enumerate}

	\subsection*{Обозначения}
	
	(const) $N$ -- количество различных ключей, на хранение которых рассчитано хранилище. (100000)
	
	(const) $S_v$ -- размер хранимых значений (в байтах). (2KB)
	
	(const) $S_k$ -- размер хранимых ключей (в байтах). (128b)
	
	(const) $F$ -- объем памяти на используемом флеш-накопителе (в гигабайтах). (128GB)
	
	(const) $E$ -- максимальное количество циклов перезаписи каждой ячейки на используемом флеш-накопителе. (10 000)
	
	(const) $U_w$ -- скорость, с которой поступают запросы на запись. (5 МБ/с)
	
	(variable) $L$ -- максимальное количество записей, единовременно лежащее в Log
	
	(const) $S_h$ -- размер хэша, распределяющего ключи по блокам (в байтах) ($\sim\lceil\frac{\log_2 \frac{N}{b}}{8}\rceil = 2\ \operatorname{or}\ 3$)
	
	(const) $S_o$ -- размер указателя на позицию в файле (в байтах) ($ \lceil\log_2N\rceil = \frac{17}{8}$)
	
	(const) $b$ -- количество пар ключ-указатель, помещающееся в один блок диска ($\lceil\frac{block\ size}{S_k + S_o}\rceil = 225$)
	
	\subsection*{Метрики}
	
	\large$WA = \frac{S_v + \frac{1}{L}\cdot(\frac{N}{b}\cdot (block\ size))}{S_v + S_k}$\normalsize
	
	$S_v$ -- запись файла
	
	$\frac{N}{b}\cdot (block\ size)$ -- размер индекса, который перезаписывается каждые $L$ записей\\
	
	\large$RA = \frac{L + 2(N-L)}{N} = 2 - \frac{L}{N}$\normalsize
	
	Вероятность найти сразу в логе -- $\frac{L}{N}$, тогда сразу считаем файл с диска, иначе 2 чтения (страница индекса и сам файл)
	
	\large$SA = \frac{N\cdot S_v + \frac{N}{b}\cdot (block\ size)}{N(S_v + S_k)} = \frac{S_v + \frac{block\ size}{b}}{S_v + S_k}$\normalsize
	
	Сами файлы и индекс

	\large$MO = \frac{\frac{N}{8} + \frac{N\cdot S_h}{b} + L(S_k + S_o)}{N}$\normalsize
	
	Битмаска свободных мест, хэши для промежутков в индексе и лог, соответственно
	
	\large$T = \frac{F\cdot E}{WA\cdot U_w}$\normalsize
	
	\subsection*{Компромиссы}
	
	Выбора тут не особо много, единственная переменная -- размер лога. Её влияние:
	
	$WA\downarrow\downarrow\downarrow\ \ RA \downarrow\ \ MO \uparrow\uparrow\uparrow$

	\subsection*{Оптимум}
	
	Подставим конкретные значения из условия:
	
	\large$WA = \frac{2048 + \frac{1}{L}\cdot(\frac{10^5}{225}\cdot 4096)}{2064} = 0.99 + \frac{882}{L}$\normalsize
	
	\large$RA = \frac{L + 2(10^5-L)}{10^5} = 2 - \frac{L}{10^5}$\normalsize
	
	\large$SA = \frac{2048 + \frac{4096}{225}}{2064} = 1.001$\normalsize
		
	\large$MO = \frac{N}{8} + \frac{N\cdot S_h}{b} + L(S_k + S_o) = 13400 + (18+\frac{1}{8})L$\normalsize
